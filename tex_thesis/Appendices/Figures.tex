\begin{landscape}
\begin{figure}[ht]
\centering
\includegraphics[width=\textwidth+2cm,height=\textheight+2cm,keepaspectratio]{images/lit_database_progrLandscape}
\caption[Illustrates a literature database that is used for reviewing R and Python ecosystem.]{Illustrates a literature database that is used in section \ref{rpythonlandscape} to provide a review of corresponding articles where surveys of different programming ecosystems have been conducted, see \path{Master_DS_DP.xlsx}.}
\label{fig-literature-programLandscape}
\end{figure}
\end{landscape}

\begin{landscape}
\begin{figure}[ht]
\centering
\includegraphics[width=\textwidth+2cm,height=\textheight+2cm,keepaspectratio]{images/lit_database_surveyTools}
\caption[Illustrates a literature database that is used for inspecting studies which survey computer programs.]{Illustrates a literature database that is used in section \ref{surveyofTools} to provide a summary of articles surveying different software applications, see \path{Master_DS_DP.xlsx}.}
\label{fig-literature-database}
\end{figure}
\end{landscape}

\begin{landscape}
\begin{figure}[ht]
\centering
\includegraphics[width=\textwidth,height=\textheight,keepaspectratio]{images/DW_BI_System}
\caption[Illustrates a typical approach to building DW/BI system.]{The figure draws upon \textcites{WarrenThornthwaite2012MicrosoftApproach}{Kimball2008TheToolkit} and presents a typical architectural design of building data warehouse/business intelligence system, see \path{DW_BI_Structure.pdf}. 
Data are first extracted from the source systems such as (non-)relational databases or flat files (for instance comma-separated values documents) into the staging area. 
Then, they are transformed for building the \emph{dimension} and \emph{fact} tables. 
At last, they are loaded for example into data marts and \ac{OLAP} cubes in order to present them in the diverse \ac{BI} frond-end applications.}
\label{fig-bi-dw-schema}
\end{figure}
\end{landscape}

\begin{figure}[ht]
\centering
\includegraphics[width=\textwidth,height=\textheight,keepaspectratio]{images/CRISP-DM}
\caption[Illustrates six phases of CRISP-DM methodology.]{Illustrates six iterative phases of \acs{CRISP-DM} methodology and the prevalence of \emph{data strategy}, \emph{data science} and \emph{data engineering}. 
All this with regard to each \ac{CRISP-DM} stage where the portion of data strategy tries to bridge the gap between \emph{what} and \emph{why}. 
On the other hand, the part of data science combines business and technology skills for answering targeted questions, for instance through model building. 
Finally, data engineering works on the end-to-end analytical platform where data are acquired, later processed by means of \ac{ETL} pipeline and presented to stakeholders.
Inspired by \textcites{PeteChapman2004CRISP-DMGuide}{AnjaliSamani2016WorkingTeams}.}
\label{fig6}
\end{figure}

\begin{landscape}
\begin{figure}[ht]
\centering
\includegraphics[width=\textwidth+2cm,height=\textheight+3cm,keepaspectratio]{images/lit_database_ds_dp_studies}
\caption[Illustrates a literature database that is used for inspecting information resources related to design patterns.]{Illustrates a literature database that is used in section \ref{ds_dp_related_research}. 
It provides a short summary of articles and internet sources that were reviewed, see \path{Master_DS_DP.xlsx}.}
\label{fig-literature-database-ds-dp-research}
\end{figure}
\end{landscape}

\begin{landscape}
\begin{figure}[h]
   \centering
   \begin{tabular}{@{}c@{\hspace{.5cm}}c@{}}
       \includegraphics[page=1,width=.5\textwidth+1cm, height=\textheight+1cm,keepaspectratio]{images/Thesis_protocol} & 
       \includegraphics[page=2,width=.5\textwidth+1cm, height=\textheight+1cm,keepaspectratio]{images/Thesis_protocol} \\[.5cm]
	\end{tabular}
\caption[Illustrates a thesis protocol.]{Illustrates how does the thesis protocol look at the very end, see \path{Thesis_protocol.pdf}.}
\label{fig-database-research-protokol}
\end{figure}
\end{landscape}

\begin{landscape}
\begin{figure}[ht]
\centering
\includegraphics[width=\textwidth+2cm,height=\textheight+3cm,keepaspectratio]{images/excel_database_of_sources}
\caption[Illustrates an excerpt from a concept matrix that is used for discovering pattern candidates.]{Illustrates an excerpt from a concept matrix that is used for discovering \acs{DS} design pattern candidates in \textbf{chapter \ref{chap:DSDP}}. 
The full matrix is provided in the supplement, see \path{Master_DS_DP.xlsx}.}
\label{fit-excel-database-sourcedata}
\end{figure}
\end{landscape}

\begin{landscape}
\begin{figure}[ht]
\centering
\includegraphics[width=\textwidth,height=\textheight,keepaspectratio]{images/R_Python_Database}
\caption[Illustrates R and Python software tools that were used for design patterns.]{Illustrates seventeen R and fifteen Python software tools that were identified and some of which were used for design pattern code examples. 
Only two R packages (\mintinline{R}/cloudml/ and \mintinline{R}/doAzureParallel/) are as of \monthdataMyOwn\@ in development on \emph{GitHub}.
See the database in \path{Master_DS_DP.xlsx} as well.}
\label{fig11}
\end{figure}
\end{landscape}

\begin{landscape}
\begin{figure}[ht]
\centering
\includegraphics[width=\textwidth,height=\textheight,keepaspectratio]{images/DS_CRISP-DM_Matrix}
\caption[Illustrates links between design patterns, tools and stages of CRISP-DM framework.]{Illustrates links between formalized \ac{DS} design patterns, identified software tools and their association with stages of CRISP-DM framework, see \path{Master_DS_DP.xlsx}.}
\label{secondDatabaseExcel}
\end{figure}
\end{landscape}

\cleardoublepage
