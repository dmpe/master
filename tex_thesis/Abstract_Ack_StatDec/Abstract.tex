\begin{singlespace}
\indent \textbf{Background:} A variety of software design patterns in the computer science are documented assisting practitioners to reuse best solutions to commonly occurring problems. 
However, lacking researchers' attention so far, design patterns were not yet formulated for a newly established \emph{data science} field which aims to extract actionable knowledge from data. 

\textbf{Objective:} From the perspective of data scientists and other engineers, the study set out to discover and understand \emph{data science design patterns}.
Consequently, by supplementing such candidates with R and Python code examples utilizing packages from their software ecosystems, the objective was to develop \emph{Data Science R and Python Toolkit Matrix} as well.

\textbf{Method:} Based on the \emph{data-driven design pattern production} methodology, the research first gathered relevant and contemporary source material. 
Later, through a general inductive approach, data were coded resulting into emergence of an exhaustive list of key themes from which a group of pattern candidates was described according to a defined template. 
The purposeful sampling of utilities from \acs{CRAN} and \acs{PyPI} repositories further contributed to developing above mentioned matrix too. 

\textbf{Results:} In this study, ten data science design patterns were formalized and \emph{Data Science Toolkit Matrix} was derived consisting of thirty-two R and Python tools deemed capable of addressing frequently occurring problems faced by practitioners when analysing data large and small. 

\textbf{Limitations:} Thesis' validity might have been hampered by the collection and interpretation of sources on which the study relied on to discover design patterns. 
To mitigate, a workshop with experts from the field could improve patterns' quality and gain their greater validation.
From the perspective of the toolkit matrix, the work focused only on tools for two-dimensional data, and hence omitted technologies for computer vision, all of which should be addressed in the future research.

\textbf{Conclusion:} The thesis has contributed to the study of design patterns by specifically focusing on data science domain while considering relevant R and Python tools from their software ecosystems.
For aspiring data scientists, this work laid out a foundational ground by providing them an overview of best practises for dealing with typical issues in data collection, manipulation and visualization.
As a result of identified design patterns and R and Python utilities, findings also help to better understand what other programming languages such as Julia need to offer and support in order to become widely used for the knowledge discovery purposes.

\bigskip
\noindent \textbf{Key Words:} Software ecosystems, design patterns, data science, R, Python, FOSS, packages

\end{singlespace}