\lettrine[lines=2]{\color{BrickRed}T}{his} chapter, being narrative and descriptive and subdivided into two main sections, is designed to clarify this work and develop an in-depth understanding of the subject matter.
Due to later sampling R and Python applications from their landscapes, a concept of \emph{software ecosystems} is first elucidated. 
Next, before concluding with a summary, it is aimed to describe in adequate detail available information on \emph{data science} and associate it with \emph{design patterns}.

%%%%%%%%%%%%%%%%%%  Software Ecosystems
\acused{IDE}
\acused{SLR}
\section{Software Ecosystems}
\label{secos}
Term \emph{ecosystem} has been introduced in the early \nth{20} century by a botanist Arthur Tansley.
With an origin in ecology, it has been used to denote \qcite{the physical and biological components of an environment when considered in relation to each other as a unit} \parencite[4]{Franco-Bedoya2017OpenMapping}.
Accordingly, many authors have confirmed that although having differences to a biological ecosystem, the software one \qcite{can be in some way mapped to phenomena in nature} too \parencites[101]{Dhungana2010SoftwareNature}{Jansen2013DefiningGovernance}.
For instance, landscape's growth and decline due to changing conditions or a need for \qcite{a continuous input of energy in the form of new development or maintenance} \parencites[98]{Dhungana2010SoftwareNature}{Mens2014}.

\marginnote{Business Ecosystem}
In 1999, James Moore has been first who extended the phrase to the \emph{business ecosystem} describing it as \qcite{an economic community supported by a foundation of interacting organizations and individuals -- the organisms of the business world} \parencites[45]{JamesF.Moore1999TheEcosystems}{Jansen2013DefiningGovernance}.
Simplified, it is a dynamic, constantly developing structure which revolves around producing goods and services to the benefit of customers and where three core characteristics have been identified.
For one, a network of various stakeholders such as distributors and producers \qcite{coevolve their capabilities and roles} together by creating leaders, niche and bridge players -- all moving towards the same vision \parencites[45]{Jansen2013DefiningGovernance}{LiYanRu2009}.
Secondly, it is an important notion of a \emph{platform} that gives a common ground for interaction between community members, examples of which are Microsoft Windows operating system, a vehicle manufactured by Ford or Eclipse integrated development environment (\ac{IDE}; \cite{JamesF.Moore1999TheEcosystems}).
Consequently, these platform holders become \emph{keystone companies} in the ecosystem because they can exercise their power whereby others will depend on them.
Lastly, \textcite{LiYanRu2009} has explained that a business setting allows synergic cooperation between stakeholders that results into pursuing shared interests together, while if acting alone they might be individually never successful.

Building\marginnote{Software Ecosystem} upon Moore's understanding, around 2000 a conception of \emph{digital business ecosystem} has emerged and this gave birth to a \emph{software} one which is considered by some to be its subset \parencites{Peltoniemi2004BusinessEnvironments}{Plakidas2017EvolutionQualities}.
Being a \qcite{very wide and arguably complex} phrase, it embodies a multitude of concepts that have been proposed by academicians \parencite[29]{Manikas2016RevisitingStudy}.
In principle, however, \acp{SECO} \qcite{are made of software vendors, suppliers, and users} who are put into a socio-economic context to interact with each other on a familiar technology that provides \qcite{possibilities for the actors to benefit from their participation} \parencites[1298]{HansenManikas2013}{Christensen2014AnalysisEcosystem}.

Since the introduction, scholars have significantly advanced the understanding of software ecosystems.
While the research took off approximately in 2005, only eight years later \textcite{HansenManikas2013} have conducted a large scale systematic literature review (\ac{SLR}) of ninety papers from the field published between 2007 and 2012 -- a fundamental study in this research \parencite{Manikas2016RevisitingStudy}.
Trying to answer questions as to how was the term defined or what were reported results, scientists have concluded that not only there has been \qcite{little consensus on what constitutes a software ecosystem} per se but also \qcite{little research (\dots) [has been] done in the context of real-world ecosystems} \parencites[1294]{HansenManikas2013}.

\marginnote{Classification}
Besides attempting to define rather nebulous term, literature has dealt with various categorization efforts.
In another significant study, \textcite{Jansen2009AEcosystems} has identified several \ac{SECO} challenges and argued that vendors should \qcite{separate ecosystems in three levels}:
%
\begin{compactitem}
    \item [(a)] software \emph{ecosystem level} (for example developing strategies and policies on how companies should cooperate in the environment to maximize their profitability),
    \item [(b)] software \emph{supply network level} (for example establishing relationships with buyers and suppliers through promoting events) and 
    \item [(c)] software \emph{vendor level} (for example forming blueprints and guidelines for product line planning, knowledge management and software extensibility).
\end{compactitem}
%
As a consequence, levels should be addressed individually in order to build, strengthen and maintain company's position and role in the engaged software landscape \parencites[1477]{Christensen2014AnalysisEcosystem}. 

On the other hand, \textcite[55]{Jansen2013DefiningGovernance} have produced and applied a classification model \qcite{to identify four different classes of software ecosystems}.
Scientists have called their factors: 
%
\begin{compactitem}
    \item [(a)] the \emph{underpinning (base) technology} (for example Apple iOS platform),
    \item [(b)] \emph{coordinators} (where it is controlled by a private company),
    \item [(c)] \emph{extension markets} (offering its users a commercial depository) and 
    \item [(d)] \emph{accessibility to an ecosystem} (paid submissions to the App Store).
\end{compactitem}
%
After governance tools have been described, authors have contributed by proposing a model for environment's health preservation and improvement \parencite{Jansen2013DefiningGovernance}. 
In fact, this area of ecosystem's health and its impact on the quality has been a significant sub-domain of \ac{SECO} that was extensively researched by studies including \textcites{Iansiti2004TheSustainability}{denHartigh2006TheEcosystem}{Jansen2014MeasuringHealth}.

\textcite{Bosch2009FromEcosystems} has further provided an important \qcite{basic categorization between ecosystems centred on operating systems, applications, or end-user programming}. 
Besides, differentiating them according to the platform \qcite{(desktop, web, mobile) the ecosystem is deployed on} \parencite[4]{Plakidas2017EvolutionQualities}.

Because \acp{SECO} are a part of information systems research, they are closely related to other areas too. 
They \qcite{include [organizational] theories and methods from a variety of different fields such as software engineering (\dots) or network analysis} \parencites{Barbosa201359}[29]{Manikas2016RevisitingStudy}.

While covering a \emph{platform-centric} perspective which \qcite{highlights technical and social aspects of a set of software projects, technical platforms and communities}, \textcite[3]{Franco-Bedoya2017OpenMapping} have put \ac{SECO} into a relationship with \ac{FOSS}.
Subsequently, the phenomenon of \emph{FOSS ecosystem} has been defined as \qcite{a SECO placed in a heterogeneous environment, whose (\dots) keystone player is (\dots) [a FOSS] community around a set of projects in an open-common platform} \parencite[24]{Franco-Bedoya2017OpenMapping}.
In their article, authors have systematically mapped the current state of \acs{FOSS} ecosystem research and have stated that even though \ac{FOSS} has been playing a strategic role in many public and private sectors, the body of knowledge about its relationship within a holistic universe is still scarce and in its infancy \parencites{Franco-Bedoya2017OpenMapping}{PetrovAarhus16}.

Although \marginnote{Key Components} definitions have varied among researchers, two fundamental features could be conceptualized across all studies \parencite{HansenManikas2013}.
On the one side, it is a network of actors like individuals and organizations with different roles, variable business structures and models \parencite{Joshua2013SoftwareChallenges}.
On the other hand, it is a common interest in using a specific technology \parencite{Christensen2014AnalysisEcosystem}.
These are then supported by establishing interactions, the connecting relationships that include promoting activities during conferences, sharing openly development resources or providing training and consulting services \parencites{HansenManikas2013}{Dhungana2010SoftwareNature}. 
All of which further grow and strengthen the ecosystem, building a sustainable environment.

Sketched\marginnote{Perspectives} out previously, in terms of \ac{SECO}'s views, it has become a commonplace to distinguish a \emph{platform-centric} angle which is technologically oriented dealing with software projects and a \emph{business-centric} one that sees the ecosystem rather as a socially-oriented \qcite{network of actors (\dots) and companies} interacting with each other \parencites[3]{Franco-Bedoya2017OpenMapping}{Eleni2017}.
For the research goals, by studying distinct definitions and explanations of aforementioned concept, the term is referred to as following:

\begin{displayquote}
\textbf{Software ecosystem} is a symbiotic relationship between a set of different actors and a collection of software projects that operate on a common technological platform as a unit. 
Hence, evolving together and sharing the same information, environment, resources and group of users and developers whose goal is to address their universal interests and needs.
\end{displayquote}

To summarize, software projects are increasingly co-developed in parallel, becoming more interdependent on each other, and thus attracting new participans \parencites{Eleni2017}{Mens2014}.
Within the forming community, resources are combined to experience the outcomes collectively by benefiting from economies of scale and diversity of members, resulting into achieving stated objectives faster.

With regard to widely used programming languages, these exhibit strong relationships between various stakeholders which manifests into the quality and maturity of their \acp{SECO}.
In the case of R and Python, both environments extending base languages have also been a critical factor and if not \emph{the one} why two formal systems have become popular in the \ac{DS} and \ac{ML}. 

%%%%%%%%%%%%%%%%%%%%%% R language
\acused{CPU}
\acused{DW}
\subsection{The R Project}
\label{rlang}
R being a \qcite{system for statistical computation and graphics} began as a research project in 1993 lead by Ross Ihaka and Robert Gentleman at the University of Auckland in New Zealand \parencites{RCoreTeam2017R:Computing}{RCoreTeam2017RDefinition}{RProject2017WhatR, DavidSmith2016OverHistory}.
Heavily influenced by two programming languages designed in the 1970s named S and Scheme, scientists have adopted a syntax of the former one and continued to maintain backwards compatibility with it. 
Additionally, investigators' aim was to modernize it for purposes of data analysis, and thus \qcite{the underlying implementation and semantics [were already] derived from Scheme} \parencites[299]{Ihaka1996R:Graphics}{RossIhaka2009TheFuture}.
For this reason, R is oftentimes called a dialect of S, a modern implementation in a new, vastly improved and enlarged language \parencite{RCoreTeam2017AnR}.

\marginnote{CRAN}
Naturally, R, formally governed by R Foundation that supports its development overseen by \emph{R Core Team}, can be extended by wrapping a set of functions (its main building blocks), documentation, data and tests into a unit of distribution called \emph{package} \parencite{DavidSmith2016OverHistory}.
As a result, this can be uploaded to software repositories such as \ac{CRAN}\footnote{\href{https://cran.r-project.org/}{https://cran.r-project.org -- The Comprehensive R Archive Network}} being the largest central extension market and from which R community can download and use developer-contributed tools.
In December 2017, because of language's growing popularity, \ac{CRAN} has reached a milestone of providing 12\,000 packages on the network with many more in development at \emph{R-Forge}, \emph{GitHub} and available for example on \emph{Bioconductor} that focuses only on utilities for genomic data \parencites{DavidSmith2017CRANNeed.}{German2013TheEcosystem}{Sylvia2014}{HenrikBengtsson2017Subject:CRAN}. 

\marginnote{Association with \acl{DS}}
Supporting multiple programming paradigms including object-oriented and mainly the functional style, R has become \qcite{the lingua franca of statistical computing [(\dots),] reproducible statistical research} and one of the dominant forces in field of data analytics \parencite[7]{Everitt:2006:HSA:1213890}.
In part, this has been attributed to \ac{CRAN} with a vast number of packages that provide powerful yet flexible applications, for instance for data visualizations or imputation of missing data \parencite{Plakidas2017EvolutionQualities}.
Likewise, being an interpreted language that can interface complied ones (C++ or Fortran), R executes various operations on data sets by storing them in \ac{RAM} instead of on a hard-disk \parencites{Kane2013ScalableData}{Hornik2016FrequentlyR}{DavidSmith2013AR}.
Last but not least, R specifically targets scientific and statistical computing and having its roots in the academia, it is being used across various disciplines \parencites{GarrettGrolemund2017RData}{TeachingDS2016}. 
Among others, in the biostatistics, finance and psychometrics.

Additionally, even though not being a novelty compared to other languages, in the \emph{interactive mode} user types in various R expressions into its command line shell (R console), where they are parsed and evaluated by the interpreter and results are immediately returned (so-called read-evaluate-print-loop; \cite{SmithDavid2009159}).
This allows to quickly iterate on manipulating the data set and applying various functions to it.
Although \qcite{by default R is command-line driven}, it naturally supports the \emph{batch mode} as well where R scripts can be created and executed in \acp{IDE}, \emph{StatET} or \emph{Visual Studio} to name a few \parencite[29]{Pabinger2014}. 

\marginnote{Dynamically Typed Languages}
Together with Python, both languages are \emph{dynamically-typed} where it is not necessary to indicate if a variable is of a data type integer or a string, see listing \ref{lst:dynamicstatictyping} for an example \parencites{PythonCoreTeam2012}{Lutz2013}.
Indeed, because typing is linked with a value rather than a variable, developers can change variables to diverse types \parencite{NinaBookR2014}.
On the contrary, \emph{statically-typed} languages require indicating data types and once a variable has been set to a specific type, it cannot be changed due to being associated with a variable rather than the value.
The reason being mentioned is that the choice of a typing system has important consequences for the programmer, prominently the beginner \parencite{JakeVanderPlas2016PythonHandbook}.
While the dynamic typing allows programs to be more flexible and compact, and thus these languages are perceived easier to start with, a disadvantage is having some evidence suggesting it negatively affects the runtime efficiency because of a type interference and prevents detection of programming errors by reason of decreased readability \parencites{Ousterhout1998}{FieldCadyDSBook}.

R\marginnote{Drawbacks} is nowadays used in many research fields as it claims to have a mature ecosystem of packages, which are \qcite{developed primarily by non-software engineers}, and support provided by keystone players -- notably RStudio Inc.\ that develops arguably the most prominent and homonymously named \ac{IDE} \parencite[244]{German2013TheEcosystem}.
Nonetheless, there are two important shortcomings that various groups of practitioners often talk about.
Being created and maintained largely by statisticians and mathematicians, it is considered to be a domain specific formal system where in addition commonly cited weakness is difficult to pick-up and somewhat inconsistent syntax, resulting into having a steeper learning curve \parencites{LearningCurveR2010}{HarderToLearn2017}.
Moreover, R and Python are both -- in their native implementations -- single-threaded programming environments \parencite{SimonBGDRAMR2016}.
Therefore, even though third-party alternatives such as Microsoft R Open or packages exist supplementing some parallel functionality and taking advantage of multi-core central processing units (\acp{CPU}), there are still limitations when compared to C++ and its multithreading capabilities \parencite{limrperf2015}.

%%%%%%%%%%%%%%%%%% 	Python
\subsection{The Python Language}
\label{pylang}
Python was conceived by Guido van Rossum in the 1980s when he was working at the Dutch research institute gaining his first experience with a programming language called ABC \parencites{Perez2011Python:Computing}{BillVenners2003TheRossum}.
By advancing his knowledge during its development and feeling inspired by others, a decision was taken to apply some design principles when creating a new one \parencite{TimPeters2004ThePython}.

\marginnote{Vision}
The goal was set to develop a formal system which would \qcite{serve as a second language for people who were C or C++ programmers} along with focusing on code readability, software quality and developer productivity \parencites{BillVenners2003TheRossum}{Sanner1999Python:Development}{Lutz2013}.
As a result, Python's syntax is \qcite{a remarkably simple and elegant} to follow making the language overall easy-to-read and easy-to-write \parencites[3]{Sanner1999Python:Development}.
To guide the core team during its further improvements, many of these design principles were later summarized in the philosophical text named \emph{Zen of Python} \parencite{TimPeters2004ThePython}.

Being specified in the \emph{Python Language Reference}, several compatible implementations have been developed with \emph{CPython}, written in C, being a referenced one as well \parencites{PythonCoreTeam2017Python:Language.}{PythonCoreTeam2017TheReference}.
Additionally, there are for instance \emph{PyPy} or \emph{Jython} -- where the former is focusing on performance providing just-in-time compiler and parallelism while the latter \qcite{compile[s] Python source code (\dots) to provide direct access to Java components} allowing Python to \qcite{script Java applications} as much as CPython allows to do the same with C(++) \parencite[34]{Lutz2013}.

Even\marginnote{Evolution} though first publicly released in 1991, only starting with version 2.x published in 2000 Python has grown to become a popular programming choice by being supported through \acp{IDE} and nowadays covering many use cases in the computer vision, Internet of Things and offering

\begin{listing}[H]
  \begin{minted}{R}
# in R (and in Python - not shown here), types are dynamically inferred
> exA <- 5       # assign an integer value to a variable
> print(exA) 
[1] 5
> exA <- "hello" # overwrite an earlier value with a string 
> print(exA) 
[1] "hello"
  \end{minted}
  \begin{minted}{java}
// in Java, types need to be declared before a variable is bound to value
public class StaticTyping {
 public static void main(String[] args) {
  int exB = 6;   // declare and assign an integer variable a value
  System.out.println(exB); 
  exB = "a car"; // when compiled, an error is raised because variable was declared as an integer
 }
}
  \end{minted}
\caption{Example of dynamic typing in R and static typing in Java.}
\label{lst:dynamicstatictyping}
\end{listing}

\noindent an array of web development frameworks.

In comparison to R, over the time the language has also evolved more significantly manifested by addition of new high-profile features.
However, being able to afford them, a backwards incompatible version 3.x had to be released in 2008 \parencites{GuidovanRossum2009APython}{Lutz2013}.
Its objective has been to get back closer to the \emph{Zen of Python} by eliminating design flaws and removing duplicate ways of programming, thus making only \qcite{one obvious way to do it} \parencites{BerndKlein2017HistoryPython}{TimPeters2004ThePython}{Lutz2013}.
Although a native 2to3 translation tool has been available to ease the migration pain, for some time the community was split due to many developers resisting to refactor their programs for what many have considered to be a new language \parencites{Raschka2016Python:Learning}{PythonCoreTeam20172to3Translation}{Lutz2013}.

\marginnote{General-Purpose Language}
A major distinguishing characteristic from R is that Python together with C(++) are consider to be \emph{general-purpose languages} \parencite{Cass2017,TeachingDS2016}.
R targets out of the box only the analytical domain with a substantial focus on statistical analysis. 
Hence, it does not offer any exemplar toolkits that support web development with \ac{HTML}, Cascading Style Sheets and JavaScript -- with an exception of RStudio's \emph{Shiny} framework. 
Yet its aim is to only help interactive data analysis and it cannot be used to build large-scale web applications that Python enables to create by using \emph{Web Server Gateway Interface} and its compatible packages akin to \emph{Django}.

Besides having capabilities for development of web, gaming applications or system-level scripting, Python community has developed a rich ecosystem for numeric programming as well \parencite{PythonCoreTeam2017GraphicFAQ}. 
Therefore, the language has been increasingly used these days for a scientific computing. 
Particularly, it was adopted for artificial intelligence applications such as \emph{TensorFlow} as their \enquote{first-class citizen} becoming de facto a \enquote{go-to} utility for working with multi-dimensional data \parencite{PDLMLRaschka2016}.
Similarly supporting multiple programming paradigms including primarily object-oriented style with some elements of the functional one too, it has found its place in many organizations covering multiple use cases and objectives \parencite{RamlhoLucFPython2015}.

A\marginnote{Obstacle} downside of Python is that its execution speed may not be the fastest when compared to low-level languages like Fortran \parencite{Lutz2013}.
On the other hand, R and Python allow calling a complied and optimized C(++) and Fortran source code either through native approaches or third-party packages -- for instance \emph{Rcpp} for R and \emph{Cython} for Python. 
As a result, developers can combine advantages of low-level and high-level programming. 
Whereas the former allows engineers to be close to the hardware speeding up compute-intensive calculations, the latter one permits to experience the ease of programming and use -- both of which have contributed to languages' popularity in the \ac{DS} universe \parencites{Perez2011Python:Computing}{JakeVanderPlas2016PythonHandbook}.

In the same way to R, the Python Software Foundation maintains \ac{PyPI}\footnote{\href{https://pypi.org/}{https://pypi.org -- The Python Package Index Warehouse}} which is currently serving over 130\,000 packages on its network.
Even though it has been further beyond the scope of this work to compare in depth two languages and possibly their ecosystems, a brief overview with essential characteristics is presented in Table \ref{tab:RvsPython} and next section.

\subsection{Summary of Previous Research}
\label{relResearch}
\acused{GUI}
\subsubsection{R and Python Landscape}
\label{rpythonlandscape}
While some studies from the beginning of section \ref{secos} are considered pioneering in the field, they have only dealt with \ac{SECO}'s theoretical conceptualization, modelling and categorization \parencite{Christensen2014AnalysisEcosystem}.
To better understand two programming ecosystems, a closer practical perspective was taken through an inspection of several relevant articles, see Figure \ref{fig-literature-programLandscape}.

\begin{sidewaystable}
    \begin{tabular}{|C{5cm}|C{0.37\textwidth}|C{0.37\textwidth}|}
    \hline
    \theadCenterText{Factors} & \theadCenterText{R (in version 3.4.4)}  & \theadCenterText{Python (in version 3.6.4)} \\ \hline
    Initial public release -- Foundation & 1993 -- R Foundation in Austria since 2002 & 1991 -- Python Software Foundation in the United States of America since 2001 \\ \hline
    License & GNU General Public License v2.0 & Python Software Foundation License (BSD-style) \\ \hline
    Language paradigms & \multicolumn{2}{|c|}{Procedural, functional, object-oriented and imperative, see \textcites{FieldCadyDSBook}{PDLMLRaschka2016}} \\ \hline
    Object-oriented system & 3 natively: S3, S4, RC (Reference Classes) and others via packages such as \mintinline{R}/R6/ and \mintinline{R}/R.oo/ & Only one \enquote{default} object-oriented system \\ \hline
    Compiled or Interpreted Language & \multicolumn{2}{|c|}{Both are considered to be interpreted languages, executing source code line by line} \\ \hline
    Level of Abstraction & \multicolumn{2}{|c|}{High-level languages, with an ability to interface low-level ones, mainly C(++) and Fortran} \\ \hline
    Focus/Target Audience & \emph{Domain-specific}: its core focus lies on domains of mathematics, statistics and analysis of data due to being traditionally used in the academia and research & \emph{General-purpose language}: used by a wider audience because of covering a variety programming use cases including web \& graphical user interface (\ac{GUI}) development, system-level scripting or scientific computing \\ \hline
    Code blocks -- control flow defined by & Curly-brackets \{ \} & Indentation (\enquote{off-side rule}) \\ \hline
    Support for 64-bit integers & Not natively, some limited support through \mintinline{R}/bit64/ package & Native support (\enquote{int} datatype) \\ \hline
    Support for two-dimensional data structures & Natively \mintinline{R}/data.frame()/ and \mintinline{R}/matrix()/ & Natively only limited \mintinline{python}/lists = []/, \mintinline{python}/tuples = ()/ and \mintinline{python}/dictionaries = {}/, thus packages like \mintinline{python}/numpy/ or \mintinline{python}/pandas/ are necessary \\ \hline
    Package format and installation & One common format supported natively \mintinline{R}/install.packages(pkgs = 'car', ...)/ & \emph{Egg} used by \enquote{easy\_install} or newer and natively supported \emph{Wheel} by \enquote{pip} \\ \hline
    Extension markets (besides \emph{GitHub} and similar) & \ac{CRAN} and \href{https://www.bioconductor.org/}{Bioconductor} & \ac{PyPI}, \href{https://conda-forge.org/}{conda-forge} \\ \hline
    Available implementations (besides the referenced one) & \href{https://mran.microsoft.com/}{Microsoft R Open}, \href{http://www.oracle.com/technetwork/database/database-technologies/r/r-enterprise/overview/index.html}{Oracle R Enterprise} & \href{https://pypy.org/}{PyPy}, \href{http://www.jython.org/}{Jython}, \href{http://ironpython.net/}{IronPython}, \href{https://www.activestate.com/activepython}{ActivePython} \\ \hline
    \end{tabular}
    \caption{Presents a high-level overview of two programming languages and how can they be characterized.}
  	\label{tab:RvsPython}
\end{sidewaystable}
%%%%%%%%%
Notwithstanding\marginnote{Data Sources} scholars' best effort in collecting available quantitative data for instance on a number of downloaded \ac{CRAN} and \ac{PyPI} packages which would be used to gauge a popularity, health and growth of the ecosystem, \emph{reliable} statistics have been hard to acquire for both platforms.
In fact, a lack of historical data from various open source extension markets has contributed to little research analysing other programming languages as well.
A thorough search in \emph{Google Scholar} and \emph{Scopus} databases, in addition to inspecting a sample of \ac{SECO} literature, has revealed that only Ruby with \mintinline{HTML}/RubyGems.org/ has been extensively looked at in the academia, see \textcite{Syed2013OnEcosystems}.
At the same time, examinations of software environments such as Java or Node.js have also been published \parencite{Eleni2017}. 
Though, they have been largely a domain of blog posts and magazine articles which have used them to answer how has a specific language been popular compared to others \parencites{ErikWittern2016AnNpm}{KristianPoslek2015OverviewJavaScript}{DavidStackOverflow2017}{CharlesHumble2012OracleEcosystem}.

For analysing R, the situation has been complicated by the fact that only one \ac{CRAN} mirror, created by a keystone player RStudio, can provide anonymized data about package downloads from October 2012 onwards \parencite{HadleyWickham2013TheMirror}.
However, public statistics to more than forty other mirrors are not available due to maintainers' lack of interest, capacity to provide them or simply accessible technology.
In comparison, Python community and \ac{PyPI} have renewed offering download figures only since January 2016 through means of Google's \emph{BiqQuery} warehouse \parencite{DonaldStufft2016PubliclyStatistics}.
Consequently, both data sources have been used as a proxy to gain an initial insight into R and Python userbase.

While\marginnote{Examination of R} some data have in recent years become ready to use, there is nonetheless a paucity of empirical research analysing software ecosystems and tools within them.
R's landscape, and what \textcite{MattAsay2016ExponentialCompetitors} has claimed to be a super-linear or an exponential growth of contributed \ac{CRAN} packages, has been studied by several scientists including \textcites{German2013TheEcosystem}{Plakidas2017EvolutionQualities}.

Specifically, the former have focused on understanding \qcite{code characteristics and dependency relationships between the R platform and externally developed} programs \parencite[90]{K.Plakidas2016HowEcosystem.}.
They found that not only there have been differences in the frequency of releases and maturity of different sets of packages but similarly in the quantity and quality of their documentation \parencite{German2013TheEcosystem}.
At the same time, they have observed that all libraries have been well maintained, their size was stable across above-mentioned types and additionally there has been a persisted drift towards having fewer dependencies too.

Likewise, building upon a previous study of \textcite{K.Plakidas2016HowEcosystem.}, a recent quantitative analysis by the same author has examined the \qcite{evolution of the R ecosystem's dependency network and the contribution and collaboration patterns of the developer community} through \ac{CRAN} and Bioconductor metrics \parencite[3]{Plakidas2017EvolutionQualities}.
Moreover, they have investigated environment's health concluding that increasingly, as R's adoption has grown, newcomers have emerged \qcite{display[ing] a \enquote{selfish} behaviour, feeding off the ecosystem without much involvement on their part} \parencite[62]{Plakidas2017EvolutionQualities}.
Overall, by comparing two main repositories, authors' analysis has shown \qcite{a broad but relatively shallow network, dominated by a few \enquote{big} actors that supply fundamental extensions, and which once established retain their position} \parencite[62]{Plakidas2017EvolutionQualities}.

Python's ecosystem\marginnote{Examination of Python} has been to date studied only by \textcite[13]{Hoving2013Python:Ecosystem} who have focused on the \emph{supply-network level} attempting to investigate \qcite{which characteristics [could be] identified within} the \ac{FOSS} ecosystem when examining Python's universe. 
Through analysing \ac{PyPI}, they have described attributes such as \ac{SECO}'s growth and evolution over the time and compared their results to a similar study done by \textcite{RubyJaapJansen2011} about Ruby.
Additionally, scholars have discovered significant relationships between different groups of developers (\enquote{lone-wolfs} and \enquote{one-day flies}), third-party packages (\emph{Eggs}) and communities of users \parencite{Hoving2013Python:Ecosystem}.

When looking at other reports and at the same time updating a study by \textcite{HansenManikas2013}, in a longitudinal work \textcite[2]{Manikas2016RevisitingStudy} has extended theoretical and empirical knowledge by \qcite{analyz[ing] 231 papers from 2007 until 2014} from the field.
Going into a greater detail, the author has categorized available literature where various software environments have been discussed concluding that research into \acp{SECO} has noticeably matured over the years, \qcite{both in volume and empirical focus} that has contributed to a deeper understanding of this concept \parencite[2]{Manikas2016RevisitingStudy}.
Furthermore, it has extended itself to other areas including previously mentioned \acs{FOSS} too.
Nonetheless, even though there has been a considerable amount of studies showing \qcite{signs of maturity}, it has been argued that they have lacked specific \qcite{theories, methods and tools (\dots) to software ecosystems} as well \parencite[28]{Manikas2016RevisitingStudy}.

\subsubsection{Surveys of Tools}
\label{surveyofTools}
Because\marginnote{Importance} of later having a purpose to review available R and Python libraries for \ac{DS} design patterns, another decision was taken to investigate existing studies that have conducted (qualitative) surveys of software applications.
The goal was not only to become familiar with a process of gathering and analysing tools but also developing a critical look on what was done in the past and could have been improved when carrying out similar research in this work. 

Predominantly, diverse programs have been assessed by academicians in domains of biology and computer science, see Figure \ref{fig-literature-database}. 
Within the latter group, due to a complexity and size of big data and data mining, a particular focus has been put on providing practitioners an overview of different \ac{ML} frameworks, storage systems, processing engines or applications for orchestration of server clusters. 
Notably, \emph{Hadoop} and its related toolbox has been thoroughly studied. 

However, long before \emph{Hadoop} was developed, already in 1999 Goebel and Gruenwald have conducted a review of knowledge discovery software.
They have identified forty-three software applications and discussed numerous challenges that are necessary to be addressed including having a seamless integration with databases or providing extensibility to offered algorithms.

Similarly, \textcite{Mikut2011} have surveyed 195 \ac{FOSS} and commercial programs according to seven criteria and have categorized them into nine types ranging \qcite{from (\dots) data mining suites, (\dots) business-centred data warehouses [(DW)]} and to what they have called \emph{solutions} -- technology for a specific purpose or a field \parencite[11]{Mikut2011}.
Scholars have concluded by suggesting that although there has been \qcite{a wide range of software products} covering many different use cases, \qcite{generalized mining [utilities] for multidimensional datasets such as images and videos} had been missing so far \parencite[11]{Mikut2011}.
Although a variety of libraries have been developed since then, not coincidentally, the previous factor was a motivating reason for focusing only on the two-dimensional data format as outlined in section \ref{delimitations} too. 

Once \emph{Hadoop} has arrived in 2006, an extensive examination by both practitioners as well as scholars has been conducted into its ever-growing family of big data open source software.
From recent times, \textcite{Liu2014} have investigated real time processing systems and compared their architectures and use cases. 
On the other hand, \textcite{Lara2015BigData} have put their attention on the ecosystem of frameworks for \emph{\ac{ML}} whereby this term has been associated with applying \qcite{induction algorithms to data} which allow computers to learn from past patterns and by that improve future predictions \parencites[357]{FosterProvost2013DataThinking}. 
Consequently, scientists have evaluated four applications according to criteria such \emph{scalability}, \emph{speed} and \emph{coverage} of different classes of algorithms.
Before identifying potential areas for improvements, they have \qcite{assigned a rating to each of the four} utilities in a qualitative manner based on their \qcite{exposure to each [application] and related [literature] works} -- as they have not conducted any experiments \parencite[27-28]{Lara2015BigData}.

Even though primarily targeting data analytics and processing of big data with \emph{Hadoop}, naturally surveys of programs have been conducted in other closely-interlinked domains too.
For instance, \textcite{Acetp2013} have surveyed commercial and \ac{FOSS} cloud monitoring tools in addition to considering their properties, open issues and challenges.
Alternatively, \textcite{ThomasSterling2010} have reviewed software for network analysis in molecular biology.
Furthermore, visualization programs for life sciences have been studied by \textcite{PavlopoulosBio2008} and \textcite{Pabinger2014} as well. 

%%%%%%%%%%% 		DS + DP
\acused{ETL}
\acused{SQL}
\acused{CRISP-DM}
\acused{KPI}
\acused{OLAP}
\section{Data Science}
\label{termsDef}
After introducing the concept of software ecosystems including R, Python and related research, it is necessary to narrow down and put into a relationship two critically important terms -- \emph{data science} and \emph{design patterns}.
Therefore, to begin this segment, an evolution which has led to \ac{DS} being a \qcite{hot career choice} and the \qcite{sexiest job of the \nth{21} century} is provided first \parencites{Davenport2012DataCentury}[51]{Provost201351}.

%%%%%%%%%%% 	"BI"
Discussing\marginnote{Historical Roots} data science, a frequently arising topic these days is its origin and how it was born to be a new discipline and a job position in many enterprises.
While practitioners as well as scholars such as \textcites{Provost201351}{Carbone2016ChallengesPerspective} have argued that a newly established term stands on its own to realize its full potential, others like \textcites{Larson2016AScience}{Vasconcelos2017} have seen \ac{DS} evolving from business intelligence and being a next step in the data-driven decision making.
Not coincidentally is \ac{DS} considered by some to be under the same umbrella of the \ac{BI} which dates back to 1958 when Hans Peter Luhn has for the first time introduced a concept of this system.
According to \textcites{DavidRostcheck2016DataCareer}{Chen2012}, both ultimately refer to gaining actionable knowledge by analysing raw data, though from somewhat different perspectives.

\textcite[314]{5392644} has defined\marginnote{\acl{BI}} \ac{BI} as \qcite{an automatic system (\dots) to disseminate information to the various sections of any industrial, scientific or government organization}.
Even though in the past sixty years a precise definition of \ac{BI} has proven to be elusive as manifested in the study of \textcite{Al-Eisawi2012BusinessReview}, a general understanding of it has been given in the late 1980s \parencite{Tutunea2015BusinessOverview}.
At that time Howard Dressner has attributed it to \qcite{concepts and methods to improve business decision making by using fact-based support} \parencite[176]{Negash2008}.
Thereupon, the associated technological means have led to become strategically and tactically important part of reaching informed conclusions across all company levels and industries \parencite{ArnottDAvidBIPat2017}. 

Since the end of 1960s, \ac{BI} has also gone through multiple advancements and saw \qcite{a progression in the development of such information systems} that support business activities and enable decision-makers to gain a business value from the data \parencite[342]{Shollo2016TowardsKnowing}.
Initially, managers have commonly used management information systems \parencites{realTimeBIContinental2006}{Shollo2016TowardsKnowing}{ArnottDAvidBIPat2017}.
These produced \qcite{standardised, (\dots) period reports that did not allow (\dots) on-line queries} and there was no integration between different data sources \parencites[342]{Shollo2016TowardsKnowing}{Larson2016AScience}. 
Although authors have identified and described \emph{generations} differently, two periods of time can be nonetheless clearly observed \parencites{Brian2015Gent}{Kohl2016}. 
Both of them have led to professionals nowadays demanding interactive and personalized reports available on their mobile devices to better understand customer behaviour and make predictions about future sales in the real time \parencites{Wilkerson2016CISB5941}{Boris2015}{2015arXiv151103085K}.

%%%%%%%%%%% 1+2
\paragraph*{First and Second Generation}
During 1980s and 1990s, \ac{BI} 1.0 was characterized by deploying solutions on-premise, at company's servers, where such tool-centric systems were owned, maintained and further developed by \ac{IT} departments \parencite{Kohl2016}.
Unfortunately, they were responding to business requirements by providing the knowledge and wisdom about its operation in an inadequate fashion, and thus added little to no agility which company undertakings really needed \parencite{Chen2012}.

The back-end\marginnote{Single Source of Truth} and foundation of \ac{BI} 1.0 system has been laid out in the \ac{DW} which has finally allowed to store data from heterogeneous and legacy \ac{IT} sources producing \qcite{a central repository with integrated [and cleaned] data for analysis} (\enquote{getting data in}), see an architectural sketch of such system in Figure \ref{fig-bi-dw-schema} \parencites[704]{Larson2016AScience}{Hejdanek2016NavrhIntelligencequot}{WatsonBiDatagettingIN2007}.
As a result, it has unified a way of extracting, transforming and presenting data to business users and has established a single source of truth acting as a \qcite{primary [origin] for BI information} in the company \parencites{Heinze2014HistoryIntelligence}{Negash2008}{WarrenThornthwaite2012MicrosoftApproach}.

Yet \ac{BI} front-end applications such as portals were complex and inflexible to manage even by \ac{IT} professionals, nota bene by managers and other end-users \parencites{Collier2011}.
Moreover, the industry has been dominated by conglomerates akin to SAP or Oracle who brought with them a vendor lock-in due to solutions being proprietary, expensive and demanding to implement, maintain and use for enterprises \parencite{Joly2016}. 
All this culminated into many projects turning out to be unsuccessful, not being used and ultimately abandoned \parencite{Collier2011}.
At the same time, even if they were implemented, employees could only create limited reports and conduct narrow exploratory data analyses.
In essence, they had no capability to make advanced predictions by applying statistical models as these were not widely available and easily exploitable.

When \ac{BI} 2.0 has arrived in early 2000s, it has started to embrace a new platform, namely the web \parencite{BI2.0Ov2012}.
Whereas \ac{BI} 1.0 has been dominated by analysing structured, internal information stored in relational databases, the advent of web brought semi-structured and unstructured data generated for example on social media \parencite{realTimeBIContinental2006}.
This gave rise to \qcite{a new class of [non-relational] databases known as NoSQL} offering new processing, storing and querying paradigms to real time data \parencites[5]{Davenport2013Analytics3.0}{loukides2011data}.

In addition,\marginnote{Self-Service BI} observing a market opportunity, vendors have begun focusing on business employees allowing them to tell engaging stories around user-friendly, interactive dashboards and creating actionable reports coupled for instance with geospatial data (\enquote{getting data out}; \cite{Firtik2017VizualizaceNastroju, Hejdanek2016NavrhIntelligencequot, loukides2011data, WatsonBiDatagettingIN2007}).
By offering self-service \ac{BI} programs and targeting non-experts who could complete their work without the interference from technical personnel, it has enabled business users to become less dependent on \ac{IT} units \parencite{Heinze2014HistoryIntelligence}. 
Leading to making the entire process more agile and business-centric instead of IT-centric \parencite{Joly2016}.

All this prompted to exploring data in a new fashion, making statistics more attractive and visual for end-users, and therefore providing them better than just simple static reports \parencite{IndBIHistory2013}.
Similarly, as the volume of collected data from the web continued to rise, analyses have been enhanced by the introduction of \emph{cloud computing} which permitted companies to move from limited on-premise hardware to flexible and instantaneously deployable, on-demand services \parencite{BI2.0Ov2012}. 
Moreover, studying data has been moving from purely historical to more experimental and increasingly -- by using advanced statistical models -- predictive one as well, attempting to answer questions such as \enquote{what if} and \enquote{why} \parencites{Chen2012}{Boris2015}.

However, even larger potential of \emph{big insights} through the richness of available data was not sufficiently realized and delivered on -- resulting into remaining an ongoing struggle gaining actionable knowledge, unresolved with \ac{BI} 2.0 \parencite{Kohl2016}.
Even though acquisition costs for \ac{BI} solutions have decreased over time because of the competition from new entrants and availability of cloud solutions, the vast array of gathered information has not yet been fully transformed into bringing a valuable understanding to business operations \parencite{Chen2012}.

Seen as a transition, \ac{BI} 2.0 has stipulated creation of new user-friendly tools while also looking increasingly into the future as opposed to just into the past.
Indeed, for some time it was considered as \qcite{the latest in a long line of technologies that have been developed} for the same purpose of \qcite{creat[ing] knowledge useful for decision-making} \parencites[341-343]{Shollo2016TowardsKnowing}{Boris2015}.
Albeit difficult to pinpoint to a specific year, since approximately 2010 new buzzwords such as big data or \ac{ML} have emerged and these have further accelerated changes in the \ac{BI} landscape \parencites{Chen2012}{Larson2016AScience}.

\subsection{The New Era of Analytics}
\label{dssection}
\marginnote{Genesis}
According to \textcite[10]{DavidDonoh2015Years50}, the origins of data science could be traced all the way back to 1962 when a mathematician John Tukey noted \qcite{that something like today's Data Science moment would be coming}. 
Decades later, after William Cleveland in 2001 and since 2008, D.J.\ Patil with Jeff Hammerbacher have become credited with describing and giving the term a complete meaning \parencites{Jifa2014DataScience}{Davenport2012DataCentury}{DavidDonoh2015Years50}{NinaBookR2014}.
Principally, they have all argued that the goal of science of data is finding interesting and robust patterns with great predictive powers that could be practically useful by incorporating \ac{ML} and computational statistics \parencites{Dhar:2013:DSP:2534706.2500499}{Vasconcelos2017}.
Hence, finally delivering on \emph{big insights} through utilizing scientific methods of inquiry to understand the vast array of gathered information -- \qcite{the realm of data science} \parencite[1]{FosterProvost2013DataThinking}. 

\textcites{LongCao2016}{CaoLong2017} has described that \ac{DS} has evolved from a paradigm shift in statistics which occurred when \enquote{elementary} analysis has transformed into advanced data analytics.
Building on top of that, data \qcite{analysis has shifted from descriptive (\dots) to predictive and prescriptive} one \parencites[708]{Larson2016AScience}{Provost201351}.
While it has continued to draw from the traditional concept of \ac{DW}, it has started to combine data from other external sources -- being diverse in quality and type.
With a continuing increase in produced and captured information, providing \ac{DS} in the enterprises is nowadays seen as \emph{the} latest in the development of information systems, technologies, processes and employee's roles that support company's data-driven decision-making \parencites{DavidRostcheck2016DataDifference}{JelaniHarper2014DistinguishingScience}.

As mentioned previously, some authors see \ac{DS} as a natural evolution of \ac{BI} which in the corporate environments has long been foundational at supporting mostly explorative and descriptive data analysis and that has become impacted by the emergence of data large in volume \parencites{JBL:JBL12010}{Larson2016AScience}.
As a matter of fact, \textcite{Davenport2013Analytics3.0} has called \ac{BI} \enquote{Analytics 1.0} focusing on preparing data for rather short and ad hoc analyses. 
On the contrary, \ac{DS} has been viewed as \enquote{Analytics 2.0} and a step towards developing data-enriched applications and services, see Table \ref{tab:BIvsDS} too (\enquote{Analytics 3.0}; \cite{Larson2016AScience, CaoLong2017}). 

\marginnote{Meaning of Data Science}
Although attempting to introduce the umbrella term of \ac{DS}, there has been little consensus about what it actually means -- in part because of \qcite{being a relatively young discipline itself, [which includes] facets from multiple other traditional fields of science} \parencites[6]{Carbone2016ChallengesPerspective}{CathyONeil2013DoingScience}{CarlShan2015TheScientists}.
Nonetheless, even though diverse definitions have been proposed, authors including \textcites{CharlesRoe2013ThePioneers}{FrancescoCorea2016DataMyths} have defined it in similar ways. 
Exemplary, \textcite[2151]{Dichev2017TowardsLiteracy} has viewed it \qcite{as an interdisciplinary field about scientific processes and systems to extract knowledge or insights from data in various forms}.
A comprehensive review of the concept has also been provided in the series of articles by \textcites[8]{CaoLong2017}{LongCao2016} describing the domain \qcite{that synthesizes and builds on statistics}, computing and communication \qcite{to study data and its environments (\dots) in order to transform [them] to insights and decisions}. 
Accordingly, it is viewed as:
%
\begin{displayquote}
\textbf{Data Science}, combining computer science with statistics, \ac{ML} and domain understanding, is a systematic process of studying data to extract knowledge and actionable insights. 
This, in order to communicate engaging and enlightening stories to stakeholders and develop data-enriched, value-added, products and services.
\end{displayquote}

Despite\marginnote{Portrayal} the stated definition, it has been challenging to characterise such multifaceted phrase due to for example \textcite[443]{AgarwalDhar2014} arguing that many of its pieces \qcite{have been around for a long time}, in various forms, since at least 1980s \parencite{LongCao2016}. 
Additionally and importantly, other similar terms related to data mining or \ac{KDD} have not been clearly differentiated by the scholars \parencites{Ayankoya2014}{Provost201351}. 
However, particularly data mining has been commonly assumed to be one subcomponent of a more general \ac{KDD} process whereby it lies between the data transformation and data interpretation and where specific algorithms are applied for extracting the knowledge \parencite{GoebelMichGru1999}. 
On the other hand, \ac{DS} acting as a catch-all subject matter incorporates a handful of data-oriented research fields such as big data and \ac{ML} as well as data-independent ones like stakeholder communication with project management \parencite{LongCao2016}. 
\begin{spacing}{1.0}
\begin{landscape}
    \begin{longtable}{|C{2.5cm}|L{8cm}| L{8cm}|L{4cm}|}
    \caption[Presents a synopsis of a historical and a modern approach to data analysis and how can they be compared to each other.]{Presents a synopsis of a historical and a modern approach to data analysis and how can they be compared to each other. 
    Even though being different in nature, usually elements from both strategies are used at various stages of the knowledge discovery process.} \label{tab:BIvsDS} \\
    \hline
    \theadCenterText{Dimensions} & \theadCenterText{Business Intelligence/Analytics 1.0}  & \theadCenterText{Data Science/Analytics 2.0} & \theadCenterText{References} \\
    \hline
    \endfirsthead
    \multicolumn{4}{|c|}{\tablename\ \thetable\ -- \emph{Continued from previous page}} \\
	\hline
  	\theadCenterText{Dimensions} & \theadCenterText{Business Intelligence/Analytics 1.0}  & \theadCenterText{Data Science/Analytics 2.0} & \theadCenterText{References} \\
    \hline
    \endhead
    \hline
    \multicolumn{4}{r}{\emph{Continued on next page}} \\
    \endfoot
    \endlastfoot
    Years & 1980s-1990s & Since 2010 & \textcites{Chen2012}{Davenport2013Analytics3.0} \\ \hline
    View of data analysis & Retrospective \& Descriptive & Predictive \& Prescriptive & \textcites{BillSchmarzo2014BusinessDifference} \\ \hline
    Perspective & Looks backwards in the history & Looks forwards in the future & \textcites{DavidSmith2013StatisticsBI} \\ \hline
    Role of \ac{BI} analysts \& Data Scientists & Focuses on exploration of past trends -- applied mainly by business users & Uses past data to predict the future -- applied by technology-skilled employees with interdisciplinary skills & \textcites{SaintJosephsUniversity2017BusinessDifference}{IanSwanson2016DataDifference} \\ \hline
    Objectives/ Goals & Assists employees in decision-making by providing access to high-quality, historical data by means of designing \ac{DW} & Uses hybrid data to develop predictive models that could drive the business by implementing data-driven functionality, for instance recommendations & \textcites{Larson2016AScience}{DavidRostcheck2016DataCareer} \\ \hline
    Typical techniques and outcomes & Executes slice and dice or roll-up/down operations on on-line analytical processing (\ac{OLAP}) cubes to develop visualizations with key performance indicators (\ac{KPI}), standard and ad hoc reports & Through experimental \& exploratory data mining, creates predictive models and analyses. 
    Develops interactive visualizations, forecasting reports and story-telling presentations & \textcites{DavidDietrich2014BuildingTeams} \\ \hline
    Targeted questions & What has happened\dots, how much did\dots \newline $\Rightarrow$ already known questions are answered & What will happen if\dots, why\dots \newline $\Rightarrow$ helps to discover and answer new questions & \textcites{MikeMerritt-Holmes201610Intelligence}{DavidDietrich2014BuildingTeams} \\ \hline
    Data Sources & Relational databases; flat files; enterprise resource planning, customer relationship management and other (legacy) \ac{IT} systems -- all of which are integrated into the central \ac{DW} to provide a single source of truth with structured, well-defined information & In addition to the previous ones, uses externally generated data from the web which are often stored in NoSQL and graph databases & \textcites{DavidDietrich2014BuildingTeams}{DavidSmith2013StatisticsBI} \\ \hline
    Data Age & Processing and propagation of captured data across the whole \ac{BI} system (as in Figure \ref{fig-bi-dw-schema}) can take several hours due to complex \ac{ETL} processes -- users query historical data (older than 1 day) & Data processing, propagating and querying usually happens instantaneously -- users often work with real time data & \textcites{Larson2016AScience} \\ \hline
    Data Quality & Detailed, preplanned architecture of \ac{DW} is aimed to present only complete, cleaned and formatted data of high-quality & Works with data sets which are of diverse quality, hence it relies considerably on probabilities and confidence levels & \textcites{MikeMerritt-Holmes201610Intelligence}{DavidSmith2011HowIntelligence} \\ \hline
    Used applications & Most commonly acquires commercial off-the-shelf software, typically for back-end and front-end applications; \newline Uses of \ac{SQL} for querying relational databases and spreadsheets files & Prevalence of \ac{FOSS} for (No)SQL databases and \ac{ML} frameworks, though oftentimes supplemented with commercial add-ons; \newline In addition to \ac{SQL}, uses other programming languages including R, Python or Java for end-to-end manipulation & \textcites{DavidRostcheck2016DataCareer}{DavidSmith2011HowIntelligence}{Fern2016} \\ \hline
    Role of Agile analytics and methodology & Adopting agility in a diverse team has not been simple, and thus not widely practised -- largely due to a complexity of \ac{IT} systems and projects & Due to iterative approach, workflow and the speed of having early results, it is naturally predisposition to take advantage of agility with far greater extent & \textcites{Collier2011}{Larson2016AScience}{krawatzeck2013agile} \\ \hline
    \end{longtable}
\end{landscape}
\end{spacing}

To help outlining this interdisciplinary conceptualization that applies to a variety of sciences, its three fundamental components are introduced next. 
At the core, the innovative \emph{technology} has been developed by a heterogeneous \emph{team} of engineers and scientists.
Then, by following particular \emph{processes}, the group is led to iteratively establish the \qcite{data-to-knowledge-to-wisdom thinking} \parencites[72]{LongCao2016}{FrancescoCorea2016DataMyths}{Dhar:2013:DSP:2534706.2500499}{Provost201351}.  

\paragraph*{Big Data}
With arrival of web\marginnote{3Vs}, companies have been focused on acquiring competitive advantage in their industries by exploiting the availability of gathered information not only from internal transaction systems but increasingly from new external sources, too, sensors and internet connected devices to name a few \parencite{Provost201351}. 
To describe a large amount of data collected, a vague term has been coined by the end of \nth{20} century named \emph{big data} which has been associated with three key properties \parencites{Carbone2016ChallengesPerspective}{Larson2016AScience}{Jifa2014DataScience}{MauroGreco2015}. 
Namely, data have become large in \emph{volume}, large in \emph{variety} (unstructured text and video) and of high \emph{velocity} (speed at which they are captured and need to be acted upon; \cite{Fern2016,ChenMinMao2014,2014BigChallenges}).
These characteristics are also known as 3Vs with additional dimensions being added which have been most frequently \emph{veracity} (quality and trust) and \emph{value} (potential insights gained; \cite{LongCao2016}). 

Principally, the phrase refers to raw information that is too \emph{large} and mainly \emph{complex} to be stored by the traditional data management tools \qcite{within a tolerable time} -- typically where conventional applications known from \ac{BI} would work well due to keeping more consistent and structured type of data in the \ac{DW} \parencites{CaoLong2017}{Provost201351}[173]{ChenMinMao2014}.
Accordingly, to facilitate their transformation, it has been required to develop new analytical technologies such as \emph{Hadoop} and \emph{Spark}, majority of which are open source further opening up the vendor lock-in and lowering the costs of acquisition and maintenance \parencites{Provost201351}{mcafee2012big}{MauroGreco2015}. 
Hence, new processing and storage paradigms have allowed to leverage big data with advanced \ac{ML} algorithms that are applied to extract previously unknown facts. 
All that by using for instance the power of \emph{cloud computing}, and thus effectively serving audiences with better advertising or supporting manufacturing companies in the supply-chain during their forecasting needs as well \parencite{JBL:JBL12010}.

The \emph{data deluge}, a situation when complexity and volume of information is overwhelming companies to manage and make use of it, has resulted into ever greater possibilities in understanding and profiling customers in a deeper way as well as tailoring services precisely to their wishes \parencites{LongCao2016}{ChenMinMao2014}.
As a result, \textcite{Davenport2013Analytics3.0} has argued that no longer should big data just help managers in business decisions but instead take a leading role and \qcite{drive the business} itself -- through developing more valuable personalized services and products \parencite[704]{Larson2016AScience}.

Yet, despite\marginnote{DS $\neq$ Big Data} being often used interchangeably, \emph{data science} is not equal to \emph{big data} \parencite{Fern2016}.
The distinctive factor lies in the latter one focusing on managing enormous amounts of information with help of for example distributed file systems, while the former one uses it together with \ac{ML} when it tries to produce actual value from data large and small \parencite{Sean2015}. 

\paragraph*{Data Scientists}
With regard to professionals, a novel field has also brought new requirements for a \qcite{next-generation [of] quantitative analysts} who need to have a broader and a multidisciplinary skillset with analytical mind \parencites{JBL:JBL12010}[5]{Davenport2013Analytics3.0}{Provost201351}. 
However, even though many engineers are calling themselves \emph{data scientists}, because of a multitude of technologies needed to work with analysing (big) data, enterprises have been faced with a shortage of right human talent who could acquire \qcite{evidence (\dots) from data by undertaking diagnostic, descriptive, predictive, and prescriptive analytics} \parencite[73]{LongCao2016}.
All this with the aim to provide actionable insights and intelligence for business operations \parencites{Dichev2017TowardsLiteracy}{TeachingDS2016}.

\marginnote{Required Knowledge}
According to scholars, one of the reasons for such lack has been a necessary skillset where data scientists need to learn diverse technologies ranging from applying \emph{Hadoop}'s \ac{ML} frameworks in the cloud to being capable of developing predictive models using R and Python ecosystem \parencites{Lara2015BigData}{CaoLong2017}.
As illustrated by Venn diagrams of \textcites{DrewConway2013TheDiagram}{Variousauthors2015WhatAnalyst}, these specialists must have programming skills as well as a proficiency in the mathematics, statistics and business analytics \parencites{loukides2011data}{Jifa2014DataScience}{Dhar:2013:DSP:2534706.2500499}.
At the same time being creative problem solvers, curious and have excellent communication and writing skills \parencites{Provost201351}{CaoLong2017}.
Consequently, by possessing domain understanding and an ability to set up processes for transformation and integration of various sources of information, they shall be qualified for uncovering deep business insights through descriptive, predictive and increasingly the \emph{prescriptive} modelling as well.
This is used for telling stories about future courses of action and anticipate and simulate in advance what might happen, and therefore taking proactive responses \parencite{CaoLong2017}.

\textcite[6]{JurneyRus2013} has described that these\marginnote{Roles and Responsibilities} experts have a role to \qcite{explore and transform data in novel ways (\dots) and combine [them] from diverse sources to create new value}.
Ultimately, data scientists look for unknown, ask improbable questions and make explorations into diverse data sets. 
Among others, they make visualizations to expose \qcite{early and often} hidden facts and patterns around which business stories can be told and issues solved \parencite[6]{JurneyRus2013}.

It has been further noted that data scientists have diverse responsibilities during the \ac{KDD} life-cycle \parencite{CaoLong2017}.
Once understanding the objectives, they need to formulate research questions, transfer business problems into analytical tasks and subsequently understand data they intent to work with. 
When this has been collected and potentially enriched with other sources, they build complex pipelines which prepare data for applying \ac{ML} algorithms to create powerful models aiming to turn raw data into information, later into insight and at last in \qcite{business decision-making actions} \parencites{GoebelMichGru1999}[30]{CaoLong2017}.
Furthermore, the role involves typical project management duties, including careful project planning, resource allocation, reviewing requested changes,  mitigating risks and last but not least ensuring a proper project closure.
Thus, by leveraging the predictive modelling and big data, data scientists have become a centrepiece to many, these days, data-driven businesses -- be it in the finance or marketing. 
In fact, through the results of data analytics, they are competent to transfer them \qcite{into benefits for society} at large and companies alike \parencites{Dhar:2013:DSP:2534706.2500499}[6]{Carbone2016ChallengesPerspective}.

\paragraph*{Iterative, team-based development}
\textcite[705]{Larson2016AScience} have further reported that \ac{DS} \qcite{involves iterative [and incremental] development of analytical models where [they] are created, validated, and altered until the desired results are achieved}.
Hence, the importance of \ac{ML} that gives computers the ability to learn by themselves without being preprogrammed doing so \parencite{IBMMLSamuel1959}.
Because of this interactivity during which (big) data are transformed and prototypes are developed, tested and adjusted, any novel discoveries may lead into changing earlier results, thus being in constant feedback loops \parencite{GoebelMichGru1999}.

Nonetheless, even though desired, one single person often does not possess necessary skills to cover all areas of \ac{DS} which span analysis, development and deployment of solutions \parencite{Carbone2016ChallengesPerspective}. 
On these grounds, a continues collaboration and communication within usually a smaller analytical team of data engineers, product managers, \ac{UI/UX} designers and researchers is necessary for successful value-based driven knowledge extraction in enterprises \parencites{JurneyRus2013}{Domino2017DS}{CarlShan2015TheScientists}. 

According to the\marginnote{Agility} literature, when compared to the traditional \ac{BI} that has usually applied a waterfall method for development, \ac{DS} has also been better predisposition to benefit from using agile practises due to inherently practising them with far greater frequency and success \parencites{Collier2011}{krawatzeck2013agile}.
As described by \textcite{Larson2016AScience}, the nature of \ac{DS} encourages persistent interaction with stakeholders to confirm the direction and quickly respond to changes and results rather than following a strict plan. 
Moreover, the aim is to work on smaller and shorter releases while at the same time have a valuable and usable documentation -- precisely where design patterns can have a noticeable impact \parencite{Larson2016AScience}. 

While\marginnote{CRISP-DM} the technological landscape has advanced profoundly with arrival of big data, the methodological approach to \emph{data discovery}, which starts as soon as raw data have been acquired, has not changed significantly over the past decades \parencites{Larson2016AScience}{Lara2015BigData}. 
Indeed, a host of models has been developed and one of these processes \qcite{with reasonably well-defined stages} for acquiring knowledge and delivering actionable business information has been \acl{CRISP-DM} (CRISP-DM; \cite[56]{Provost201351}).
Claiming to be one of the most frequently applied and adjusted frameworks, it has been published in 1999 to service the needs of data mining community by helping them to solve common pitfalls in data-driven projects in terms of understanding, preparing and analysing their information \parencites{PeteChapman2004CRISP-DMGuide}{ThomasZeutschler2016ITAnalytics}{Firtik2017VizualizaceNastroju}{GarrettGrolemund2017RData}.

As shown in Figure \ref{fig6} and seen next, the cyclical and iterative \ac{KDD} process model consists of six phases which according to researchers such as \textcite{JurneyRus2013} should also fit all within one to four weeks of development, named \emph{sprints}:
\begin{compactitem}
    \item [(a)] \emph{business understanding} (assessing situation and determining objectives),
    \item [(b)] \emph{data understanding} (collecting and exploring data),
    \item [(c)] \emph{data preparation} (data cleaning and formatting presenting the largest effort overall),
    \item [(d)] \emph{modelling} (selecting, designing and building statistical models),
    \item [(e)] \emph{evaluation} (of results and reviewing the process) and finally
    \item [(f)] \emph{deployment} (of solutions, making use of outcomes and finalizing the project).
\end{compactitem}
Each of these stages has other \emph{general} and \emph{specialized} tasks and by following these steps, the framework providing \qcite{the overview of data mining life-cycle} should help businesses to reduce operational costs, increase time-to-market and support knowledge transfer within organizations through appropriate, domain-specific management of analytical projects \parencites[16]{ThomasZeutschler2016ITAnalytics}{Horvath2011}.
Ultimately, providing a tool-, industry- and technology-neutral model in the business context \parencites{PeteChapman2004CRISP-DMGuide}.

%%%%%%%%%%%%%%%%%
\acused{API}
\subsection{Design Patterns}
\label{dp-intro}
Attempting to put into a relationship \ac{DS} and a concept of \emph{design patterns}, the last key term is introduced. 
As briefly outlined in the introduction, in 1977, a civil engineer named Christopher Alexander has published a literature work, nowadays a perennial seller, entitled \emph{A Pattern Language} which gave a birth to a pattern movement \parencites{SalingarosSOMEALEXANDER}{DongPan1998TheEngine}. 
Being grounded in the domain of architecture, the profound work of \textcite{Alexander1977} has defined each pattern portraying a \qcite{problem which occurs over and over again (\dots), and then describes the core of the solution to that problem, in such a way that you can use this solution a million times over, without ever doing it the same way twice} \parencites[93]{Spinellis1999}. 

Patterns described in prose are useful means \qcite{of capturing time-tested[, optimized and generalized] design solutions and facilitating their reuse} -- thus helping \qcite{people reason about what they do and why} through provided terminology \parencites[853]{HeffreMheer2006}[37]{Schmidt:1996:SP:236156.236164}{BruseDougals2002}.
In the words of \textcite[13]{DongPan1998TheEngine}, they establish the ability to discuss and \qcite{record design tradeoffs}. 

Unfortunately, a single pattern may be insufficient to describe a wide-ranging challenge and there might be even alternative solutions too.
Therefore, \textcite[37]{Schmidt:1996:SP:236156.236164} has stated that \qcite{when related [context-dependant] patterns are woven together they form} a system of \emph{standard vocabulary} and effective pieces of advice to \qcite{recurring problems} in various fields \parencites[10]{Fowler2002}{InPaulPeter2016}{InventadoPeter2015}{geist2012patterns}{Wilson2008PatternsEnvironments}. 
As a result, a \emph{pattern language} such as the one for planning and construction purposes or in the \textbf{chapter \ref{chap:DSDP}} represents a network of relationships to resolve the complexity in specific situations \parencites{HeffreMheer2006}{DeardenHCI2006}.

Correspondingly, the\marginnote{Value} advantages of using systematic approaches to commonly occurring problems for instance in phases of software analysis, design and implementation are several fold. 
First and foremost, according to \textcite[93]{Spinellis1999}, design patterns \qcite{offer a convenient way to capture, document, organise, and disseminate existing knowledge from a given area in a consistent and accessible format}.
Furthermore, the use of patterns in a conscious way can improve the overall understanding of sophisticated concepts and provide a better transfer of experience within a company as part of keeping and enhancing its organisational memory \parencites{Chetan2016}{Schmidt:1996:SP:236156.236164}{DeardenHCI2006}. 
The existing body of research has also suggested that they \qcite{lower total cost of ownership} due to fostering communication of ideas, coordination and collaboration of work between stakeholders \parencites[1]{Adriadno2016}{Fowler2002}.  
Principally, however, patterns are responsible for making better design decision through using \qcite{simpler, [scalable,] more flexible, modular, reusable, and [generally] understandable} techniques, best practises and proven solutions \parencite[1]{ChenHong2004}. 

Nonetheless\marginnote{Drawbacks}, while having benefits of improving project documentation and being a \emph{lingua franca} in communication with domain experts, at the same time patterns might prolong time required for development of applications as well \parencite{Hossam2017}. 
Moreover, if used improperly, they can decrease source code understandability due to a larger complexity of the implementation \parencite{DeardenHCI2006}.
Besides, \textcite[2]{Hossam2017} has listed that they could limit design options and \qcite{leave some important details unresolved}, thus acting more like templates that demand further refinement and which cannot be blindly used \parencite{Fowler2002}. 
For these reasons, as noted by \textcite[37]{Schmidt:1996:SP:236156.236164}, \qcite{all solutions have costs, and pattern descriptions should state the[se] clearly}.
From the supplementary perspective of their characterization, \textcite[4]{PeterNorvig1996DesignProgramming} has likewise reported that because they are invisible, they are difficult to notice and formalize as it is a human who shapes them to \qcite{explicitly appeal to aesthetics and utility}.
On the other hand, once established they can become an invaluable resource for stakeholders' daily tasks.

Both\marginnote{Pattern Structure} \textcites{BruseDougals2002}{DongPan1998TheEngine} have remarked that writing a pattern consists of three core components. 
Namely establishing a relationship between a \emph{problem} it is supposed to address, a \emph{solution} which is the pattern itself describing various \qcite{elements that make up the design} and a specific domain context in which it would be applied \parencites[13]{GoF2002}{DobleMeszaros1997}. 
In addition, researchers have stated that patterns should have an evocative \emph{name} forming a common vocabulary for effectively sharing experiences and knowledge, a \emph{diagram} or \emph{sketch} for instance in the \ac{UML} and a \emph{summary} describing its purpose \parencites{Fowler2002}{HeffreMheer2006}{PeterNorvig1996DesignProgramming}.
Last but not least, there are \emph{forces}, \emph{implementation details}, \emph{consequences} of its use and \emph{examples} demonstrating the application \parencites{ChenHong2004}{Stefan2017}{GuerreroLuisFuller2001}{DobleMeszaros1997}. 

\marginnote{Application}
Not being invented but rather discovered \qcite{by trial and error and by observation}, design patterns were applied in various domains from education, e-learning and communication to over being applied for game design, sociology or in the business \parencites{Gwendolyn2010}{EducationPatterns2012}[63]{DeardenHCI2006}. 
Focusing in this work on their use in the \ac{IT}, numerous documents have already been published summarizing widely used approaches in many computer science disciplines.
Case in point, books by \textcites{Brich1995}{Mattson2005} have identified patterns for concurrent, parallel and distributed programming.
Works of \textcites{Erl2015}{Fehling2014}{Blaha2010} have dedicated their attention to cloud computing and database modelling. 
\textcite{Fowler2002} with \textcite{BobbyGregor2012} have presented related best practises when integrating and developing enterprise, service-oriented middleware applications where such patterns have an impact on the whole system due to their extensive scope \parencite{BruseDougals2002}.

\marginnote{\ac{GoF}}
However, perhaps the most important pattern-related publication has been released in the software engineering by Gang of Four authors in 1994 who popularised twenty-three patterns for object-oriented software design, categorizing them into \emph{structural} (examining relationships), \emph{behavioural} (dealing with object interaction) and \emph{creational} ones (handling object formation; \cite{DongPan1998TheEngine}). 
Having a clear proposition and being local in scope, practitioners have explained in the \qcite{collection of relatively independent solutions} that their patterns describe a communication and collaboration of objects and classes \qcite{that are customized to solve [issues] within a particular [programming language independent] context} \parencites[853]{HeffreMheer2006}[39]{Schmidt:1996:SP:236156.236164}.

It has been only since Gamma's et.\ al (1994) fundamental text that the study of design patterns has gained a momentum in the research and has extended itself into other areas of human-computer interaction including \acs{UI/UX} design \parencites{DongPan1998TheEngine}{DeardenHCI2006}.
This effort has been also spearheaded by \emph{The Hillside Group} through which participating scholars have developed an extensive body of knowledge that is being shared on \emph{\ac{PLoP}} conferences.
Indeed, given their importance for the community, both information sources have been an instrumental gateway in this understanding of design pattern research.

When\marginnote{Bridge Pattern} looking at a typical example described by \ac{GoF}, a structural \patternName{bridge pattern} has been illustratively implemented in the Java Database Connectivity application programming interface (\ac{API}).
Its intent has been to solve a problem of \qcite{decoupling abstraction from implementation so that the two can vary [easily and] independently}, essentially separating two concerns, and therefore improving their extensibility \parencite[171]{GoF2002}.
Hence, the goal of such pattern is assuring that there is a class separation where the developer does not call a platform- or vendor-specific implementation but instead uses its abstraction, a general and independent interface.
Besides, \ac{GUI} frameworks like \emph{Qt} have adopted it to provide a set of consistent interface elements that could be used across all supported hardware platforms while at the same time considering various specificities of operating systems. 

\marginnote{Context} 
Turning now the attention to design patterns used for \ac{DS} and encompassing domains of \ac{BI}, big data and \ac{ML}, these shall be capable of capturing \enquote{big ideas} and best practises across different levels of abstraction -- providing both theoretical and technical solutions \parencites{MosaicDataScience2017}{Chetan2016}{DeardenHCI2006}{DelibasicBKirchnerK2008AApproach}. 
As such, the discovered \emph{data science design patterns} should offer effective and efficient means to problems found in the \ac{DS} life-cycle. 
This, when following for instance \ac{CRISP-DM} and its stages of preparing, modelling and evaluating information which might be large in volume, variety, velocity and is often of a great uncertainty \parencite{Hossam2017}.
The objective of these \emph{higher-order abstractions} is to symbolize the knowledge that is implicitly understood by the experts in order to share the wisdom with other \ac{KDD} professionals and stakeholders in the organization \parencite{HeffreMheer2006}.
Thus, precisely describe a solution to an issue \qcite{in a given context}, see next \parencite[93]{Spinellis1999}. 

\begin{displayquote}
\textbf{Data Science Design Patterns} represent reusable design solutions that solve a frequently occurring problem in the interdisciplinary field which studies data by means of applying scientific methods to extract valuable and actionable knowledge and insights.
\end{displayquote} 

\subsubsection{Related Research}
\label{ds_dp_related_research}
Moving\marginnote{\acp{SLR}} on now to consider related research in the field of design patterns, several mapping studies have been conducted and one of possibly the most comprehensive ones has analysed 637 articles published between 1995 and 2015. 
\textcite{DPSummarySMS2016} have broadly classified the research into six pattern categories dealing with: 
%
\begin{compactitem}
    \item [(a)] \emph{development} (also the most frequent objective),
    \item [(b)] \emph{usage} (pattern utilization and application), 
    \item [(c)] \emph{mining} (techniques for their detection), 
    \item [(d)] \emph{quality evaluation} (assessing their impact), 
    \item [(e)] \emph{specification} (methods and notation for their description) and 
    \item [(f)] \emph{miscellaneous issues} (for example their use in refactoring).
\end{compactitem}

Earlier, \textcite[14]{GoFDesignPatternsAmpatzoglou2013} have arrived to a similar taxonomy when they have specifically targeted \ac{GoF}-related \qcite{studies that deal[t] with the effect of pattern application on quality} features. 
Consequently, they have concluded that patterns \qcite{enhance one quality attribute [for instance the functionality and usability] in the expense of another} one -- hence they \qcite{cannot be characterized as universally \enquote{good} or \enquote{bad} } \parencite[14]{GoFDesignPatternsAmpatzoglou2013}.

After having discussed two general studies that were conducted to analyse the field of pattern research, an extensive search in scholarly databases such as Scopus and Google (Scholar) has revealed that \emph{data science(-oriented) design patterns} have so far received little attention as suggested by the paucity of studies that have been dedicated to them.
According to \textcite[2]{CaoLong2017}, one possible explanation for this lack could be attributed to \ac{DS} being a broad concept and \qcite{at a very early [exploration] stage}.
Though, on the other hand, once being further narrowed down to its individual sub-fields like big data processing or prediction utilizing \ac{ML} methods, a handful of relevant studies as well internet resources have been identified and catalogued in Figure \ref{fig-literature-database-ds-dp-research} too.
Therefore, following paragraphs consider numerous subject areas, one of which is the information visualization -- the outcome of data analysis and arguably the primary focus of \ac{DS} by which stories can be told and action can be taken.

\marginnote{Visualization}
Indeed, this has already been extensively studied by scholars who have documented various approaches to \qcite{model[ing], design[ing] and perform[ing]} decorative tasks on data, see \textcites[5]{ChenHong2004}{WareColin2013}{HeffreMheer2006}.
A classical literature work by \textcite{Tutfe1883} has laid out foundational grounds and inspired both designers and programmers in incorporating visual thinking processes into common design practices.
Therefore, maximizing information value for the end-user while at the same time minimizing a development effort for software engineers or \ac{UI/UX} creators \parencite{WareColin2013}.
Specifically, \textcite{ChenHong2004} has presented nine high-level solutions for interactive visualization which have been categorized into \emph{data}, \emph{structural} and \emph{behavioural} patterns illustrated on examples of \patternName{visual encoding}, \patternName{graphic grid} and \patternName{brushing}.
On the other hand, at a lower-level \textcite{HeffreMheer2006} have proposed a pattern language consisting of twelve problem and solution pairs including \patternName{scheduler} and \patternName{camera} which have been observed in various \ac{GUI} frameworks and applications.

\textcite{MinerDonald2012} have\marginnote{MapReduce} classified over twenty design patterns into six overarching themes dedicated to \emph{Hadoop} and \emph{MapReduce} paradigm. 
Moreover, they have stated, not surprisingly and in line with findings of this study too, that research has heretofore been scarce and scattered across diverse, often unreliable blogs, websites or hidden in books and articles where a small chapter is devoted to recurring issues and their solutions in data analytics. 

While\marginnote{Big Data} \textcite{GoF2002} have been dealing with classes and objects, from an architectural point of view, patterns discovered in big data have been touching multiple different components \parencites{Fabian2013}{Fowler2002}. 
Hence, trying to understand how complementary enterprise systems have been designed, \textcite{Forrester2013} of Forrester Research have interviewed professionals in eleven companies and have uncovered four architectural concepts. 
Ultimately arguing that \emph{hub-and-spoke approach} has been providing best capabilities and extensibility through having a common low-cost storage layer to which various \ac{DW}, \ac{BI} tools or mathematical packages could be connected and later used by company's departments for their specific needs.

The on-line catalogue at \mintinline{html}/BigDataPatterns.org/ has given further insights into necessary \ac{IT} artefacts describing diverse processing and query engines and other applications that have been found within the current analytical systems in organizations \parencites{Arcitura2017BiDataPatterns}{erl2015big}.
Once \emph{technological mechanisms} have been presented, the same authors have introduced thirty-five big data design patterns which have been later grouped into a new perspective of compound patterns. 
These combine a set of stand-alone ones to create models or technology-sets -- for example \patternName{analytical sandbox} or \patternName{big data processing environment}.

Being\marginnote{Layered Architecture} known from the enterprise \ac{IT} architecture, \patternName{layered design} pattern has suggested to separate systems into logical, discrete \qcite{groupings of the functionality}, irrespective of hardware's \qcite{physical location} \parencite{MicrosoftPatterns2017}.
Analogous to authors such as \textcites{Hossam2017}{Lara2015BigData}, \textcite{IBM2013DeveloperWorks} have described four building blocks of a solution platform for insights discovery, see Table \ref{IBMtable}.
Going more in depth, it has been suggested for company employees \qcite{to think in terms of big data requirements and scope} instead of layers \parencite{IBM2013DeveloperWorks}.
Thus, \emph{atomic}, \emph{composite} and \emph{solution patterns} have been proposed. 

Addressing specific requirements, the first group has been further divided into patterns for \emph{data consumption}, \emph{processing}, \emph{storage} and \emph{access}.
On the other hand, the composite ones such as \patternName{store and explore} encapsulate several dimensions to solve \qcite{a given business problem} and together with atomic ones contribute to establishing the final category dealing with business scenarios, for instance \patternName{take the next best action} \parencite{IBM2013DeveloperWorks}. 
Overall, layers are important because it is them which programmers and researchers look at when mapping existing software tools and identifying gaps that need to be addressed by developing new technology \parencites{Fowler2002}{IBM2013DeveloperWorks}. 
As a result, they should enable better understanding of what (non-)functional requirements and tasks are necessary to perform in these contexts too.

\begin{spacing}{1.0}
\begin{table}[ht]
\centering
\caption{Presents logical layers as they have been described by \textcite{IBM2013DeveloperWorks}.}
\label{IBMtable}
\begin{tabular}{|l|l|}
\hline
\theadCenterText{Tier} & \theadCenterText{Meaning/Examples} \\ \hline
Big Data sources & Format and point of data location and collection \\ \hline
Data massaging and store layer & Distributed file systems, relational databases \\ \hline
Analysis & Recommendation engine \\ \hline
Consumption & Presentation/visualization capabilities \\ \hline
\end{tabular}
\end{table}
\end{spacing}

The\marginnote{Deep Learning} yet-to-be-published book by \textcite{PerezBook2017} entitled \emph{Design Patterns for Deep Learning Architectures} concerns a biologically inspired concept within the artificial intelligence called deep learning \parencite{ArelDL2009}.
In author's ongoing research, patterns have been categorized into seven thematic areas ranging from the \emph{representation} (\qcite{how neural networks represent data}) and \emph{explanation} dealing with \qcite{different kinds of output from neural networks} to \emph{serving} (deployment of models; \cite{PerezBook2017}).

Besides\marginnote{Data Science} another forthcoming book by \textcite{Todd2019} mentioned already in the introduction, up to now very little works have included phrases of \ac{DS} and design patterns together. 
Illustratively, \textcite{MosaicDataScience2017} has defined this compound concept as \qcite{reusable computational pattern[s] applicable to a set of data science problems having a common structure, and representing a best practice for handling such problems}. 
In their five technical blog posts, supplemented occasionally with R code examples, authors have looked at approaches that transform and combine individual variables or handle the missingness in data.

Last\marginnote{Data-Oriented Patterns} but not least, \textcite{Fabian2013} has written a chapter in his work on data-oriented software design where he listed, among others, eight specific patterns like \patternName{in place transformation} which modifies data (structures) within the same container or \patternName{tasker} that concurrently runs transformation tasks on many data pieces without considering other parts. 
Being foundational, both have been implemented in many programming languages including R and Python and within their ecosystems as their primary object of concern are raw data.

%%%%%%%%%%%%%%%%%%%%%% 
\section{Summary} 
To summarize, by pursuing a first study question, the goal of \textbf{this chapter} has been to provide a background to important concepts in this thesis.

\marginnote{SECO and R \& Python}
It has begun with section \ref{secos} which has not only allowed to gain a better understanding of two programming languages but also familiarize the audience with previous research in the field.
Consequently, its importance lies in deepening and broadening the knowledge that is further needed for answering the second and third study question. 

Albeit having strengths in different areas, one of the main reasons why two formal systems are applied in the \ac{DS} community is their ecosystem offering a wide range of cutting edge algorithms that are indispensable in order to effectively mine data for obtaining a wisdom.
As touched upon in the introduction, R and Python have been often compared to each other trying to identify which is a preferred choice to accomplish all developer's demands and use cases.
Nonetheless, it is usually required to use a multitude of tools, and therefore \textcite{Theuwissen2016} has highlighted the need to learn one programming language after another as they also share parallels in terms of orientation and functionality.

While both have been actively evolved by their core team of contributors, Python has continued to attract larger attention due to being generally purposeful and having more visible changes between the releases \parencite{Cass2017}.
On the contrary, R has aimed for a backwards compatibility all the way back to its roots, and therefore has followed a more conservative approach whereby relying on community efforts and particularly \ac{CRAN} more significantly.

Additionally, when looking to gain a deeper understanding of programming ecosystems and especially those of R and Python, section \ref{rpythonlandscape} has uncovered that research has been done largely on the \emph{software ecosystem} and \emph{supply network level}. 
In fact, taking a holistic look at their tools, conducting analyses of used packages and principally studying the \emph{vendor (developer) level} has been avoided. 
Unsurprisingly, due to a lack of necessary data and complexity of their acquisition, existing studies have primarily analysed R. 
Moreover, only the work of \textcite{Eleni2017} has attempted to examine ecosystems in combination and no other studies were found mapping available tools for \ac{DS} purposes or interconnecting them with \emph{design patterns} as well.
This is where this thesis aims to intersect between subject matters when aggregating rather specific R and Python application universes with a more generic conceptualization of \ac{DS} design patterns.

When taking a deeper look at various surveys of utilities published to date, their review in section \ref{surveyofTools} has demonstrated shortcomings when observing that only higher-level inquiries of diverse types of programs were made. 
Indeed, a strong focus has been put on \emph{Hadoop} family, meanwhile, R's and Python's specific features and \ac{KDD} ecosystem capabilities have not yet been thoroughly investigated.
Furthermore, scholars have usually inadequately explained how their surveys have been methodologically conducted in terms of tools' discovery and selection -- other than according to their defined objectives. 
This lack of details has been pointed out by \textcite{JansenHarrie2010} and as a result it is aimed to sufficiently explain how a concurrent collection of applications is done in section \ref{collectionOfTools}.

With an exception of design patterns, after investigating \ac{SECO} and \ac{DS} and attempting to understand their exact meaning, an unfortunate fact could be highlighted that none can be defined with one simple and common definition covering all aspects and views.
This is due to terms' ambiguity, often labelled as \enquote{umbrella} phrases, and thus creating a number of rationales and interpretations by researchers.
To have an overview and overcome this challenge, an attempt was made to relevantly define each concept taking into account different understandings that have been proposed in the literature.

As\marginnote{\acl{DS}} seen throughout the segment \ref{termsDef}, \ac{DS} is difficult to pinpoint exactly because it is interconnected with other related fields such as \ac{BI} and statistics from which it was born and evolved into nowadays being indispensable for companies to preserve their competitive advantage in the business \parencite{Provost201351}.
Accordingly, it was defined as an overarching concept that inherently encompasses technology for big data and \ac{ML} techniques as well.
Similarly, \textcite{Ayankoya2014} have refereed to \ac{DS} as a convergence of three themes, namely \ac{BI}, advanced analytics which is typically manifested in the form of domain-specific \ac{ML} capabilities and big data.
In this work, once three typical characteristics were described -- the use of big data by a new class of professionals who create statistical models utilizing Agile principles and following frameworks like \ac{CRISP-DM} -- it has additionally allowed to illustrate and compare what \textcite{Davenport2013Analytics3.0} has called Analytics 1.0 with Analytics 2.0 in Table \ref{tab:BIvsDS}.

Subsequently\marginnote{Design Patterns}, design patterns were finally introduced and being at the core focus of this work it was attempted to put them into a relationship with aforementioned process of extracting insights by applying scientific procedures.
Taken together, the investigation has established that to date little research was found in the academic literature with regard to design patterns and their application in the \ac{DS}. 
Although some works and internet resources have touched upon related patterns in one way or the other, they were narrow in focus on one particular stage of data analysis, paradigm or for example attempted to provide a perspective by way of layers and components of big data systems.

Continuing the quest of addressing the research gap, the \textbf{next chapter} details the methodology of how patterns are discovered.